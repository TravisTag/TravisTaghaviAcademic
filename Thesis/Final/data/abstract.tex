%%%%%%%%%%%%%%%%%%%%%%%%%%%%%%%%%%%%%%%%%%%%%%%%%%%%%%%%%%%%%%%%%%%%%
%%                           ABSTRACT 
%%%%%%%%%%%%%%%%%%%%%%%%%%%%%%%%%%%%%%%%%%%%%%%%%%%%%%%%%%%%%%%%%%%%%

\chapter*{ABSTRACT}
\addcontentsline{toc}{chapter}{ABSTRACT}

\pagestyle{plain} % No headers, just page numbers
\pagenumbering{roman} % Roman numerals
\setcounter{page}{2}

\indent This thesis work is an exploration into the application of machine learning to a current problem relating to streams of aerial images, with application to the imaging systems within the Texas A\&M GEOSAT group and to the Texas A\&M AgriLife Extension Service.
The streams of images referred to are taken by a unmanned aerial vehicle (UAV) over some space of land.

Our problem relates to flight inconsistencies in the UAV.
In an ideal scenario, our UAV captures a set of images from a specified height, pointed normally at the ground, with a predetermined amount of overlap between images.
However, due to inconsistencies inherent in most flights, the UAV will occasionally tilt or swing in such a way that the images captured are not in line with adjacent images, i.e., they do not have the required amount of overlap, and their information may be blurred.
This creates a problem when attempting image stitching afterwards, as such anomalous images will be irreconcilable with adjacent images.
A prime goal then consists of designing a system which can, in pseudo-real-time, detect images which are out of line, through the use of feature extraction and machine learning, so that they can be discarded and possibly recaptured.

Our research compares and evaluates a number of image features and distance metrics, along with supervised classification learning algorithms, on the basis of their performance in this task.
After feature evaluation, pixel intensity distribution, binary robust independent elementary features (BRIEF), and blob detection were selected as image features.
% These features are compared by a number of metrics to create a set of single-valued features for classification.
The classification models chosen include logistic regression, neural networks, decision trees, and support vector machines (SVM).

After training and testing, the logistic regression model yields the highest performance, with a detection rate of 95.6\% and a false alarm rate of 7.7\%, making this a viable option for this task.
Neural networks and SVM yield lower levels of performance in detection and false alarm rates, respectively.
Decision trees offer a very low detection rate of 11.9\%, and are thus concluded to not be an ideal model for this task.


\pagebreak{}

