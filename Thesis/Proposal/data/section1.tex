%%%%%%%%%%%%%%%%%%%%%%%%%%%%%%%%%%%%%%%%%%%%%%%%%%%
%
%  New template code for TAMU Theses and Dissertations starting Fall 2016.
%
%  Author: Sean Zachary Roberson 
%	 Version 3.08.16
%  Last updated 8/19/2016
%
%%%%%%%%%%%%%%%%%%%%%%%%%%%%%%%%%%%%%%%%%%%%%%%%%%%

%%%%%%%%%%%%%%%%%%%%%%%%%%%%%%%%%%%%%%%%%%%%%%%%%%%%%%%%%%%%%%%%%%%%%%
%%                           SECTION I
%%%%%%%%%%%%%%%%%%%%%%%%%%%%%%%%%%%%%%%%%%%%%%%%%%%%%%%%%%%%%%%%%%%%%


\pagestyle{plain} % No headers, just page numbers
\pagenumbering{arabic} % Arabic numerals
\setcounter{page}{1}


\chapter{\uppercase {Introduction and Literature Review}}

\section{Introduction}

In general, this thesis deals with statistical inference problems; these problems are most often dealt with using different estimators, such as MLE, maximum likelihood estimator.
However, in order to get ideal performance 


\subsection{Brief Usage of the Template}

The advantage of using this template over the Microsoft Word templates are numerous. First, there is a lot of control granted to the user in how the document looks. Of course, you are expected to still follow the guidelines set forth in the TAMU Thesis Manual. This template takes care of the margins, heading requirements, and front matter ordering for you.


\section{Literature Review}

All requirements for theses can be found in the most recent version of the Thesis Manual, available at the OGAPS website. The Thesis Office will be happy to assist you if you have questions about formatting. Questions specific to \LaTeX\ should be directed to \texttt{ogaps-latex@tamu.edu}.

\subsection{Another Test Section}
 
Hello, is it me you're looking for?


 

